\documentclass[twoside]{bhamthesis}

\usepackage{graphicx}
\usepackage{amssymb}

\title{Solar System Explorer - An Incremental Game}
\author{Philip Prior (0920420)}
\supervisor{Hayo Thielecke}
\degree{MSc Computer Science}
\department{Computer Science}

\begin{document}
\maketitle

\tableofcontents

\begin{abstract}

This report details the design and implementation of a software engineering project to create a space exploration themed incremental game. The challenge was to introduce a degree of realism to the representation of space beyond that in similar products from the incremental games market.

The game was created using languages that made it suitable for delivery via a Web browser, using an incremental software development lifecycle and focusing on prototyping for testing during development.

A game named “Solar System Explorer” was produced, successfully incorporating several planned features including those required to meet the definition of an incremental game.

Suggestions are also provided herein as to the further development of the product beyond the timescale of this particular project.

\end{abstract}

\begin{acknowledgments}

\end{acknowledgments}


\section{Introduction}

\section{Further background material}


\section{Analysis and specification}
\subsection{Example Requirements}
\subsection{Game specific requirements}
Draw from 

\section{Design}
\subsection{Language selection}
\subsubsection{HTML 5.0}
\subsubsection{CSS 3}
\subsubsection{JavaScript ES6}

\section{Implementation and testing}
\subsection{Data structures}
\subsection{Rotation in 3D space}
\subsection{Unit testing}
\subsection{Prototype model}

\section{User interface}
\subsection{Incremental game conventions}
\subsection{General UI conventions}
\subsection{Aesthetic}


\section{Project management}

Tools, time management, meetings.

\section{Results and evaluation}

\section{Discussion}

\section{Conclusion}

\section{References}


\declaration

\end{document}